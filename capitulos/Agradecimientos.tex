% Inicio de los Agradecimientos
\begin{thanks}

A mis Padres por haberme dado las oportunidades necesarias para superarme.

A la Ing. Areli Rubio Rodríguez por su apoyo con la redacción y revisiones del documento, así como todo su apoyo brindado mientras realizaba el proyecto.

Agradezco a mi asesor el Mtro. José Fernando Estrada Saldaña por su enorme paciencia y ayuda con el conocimiento necesario para el desarrollo del proyecto, ofreciéndome su apoyo en los tiempos más difíciles y complicados de mi etapa estudiantil.

% [Sustituye este texto escribiendo tus agradecimientos.
% La sección de agradecimientos reconoce la ayuda de personas e instituciones que aportaron significativamente al desarrollo de la investigación. No te debes exceder en los agradecimientos; agradece sólo las contribuciones realmente importantes, las menos importantes pueden agradecerse personalmente. El nombre de la agencia que financió la investigación y el número de la subvención deben incluirse en esta sección. Generalmente no se agradecen las contribuciones que son parte de una labor rutinaria o que se reciben a cambio de pago.

% Las contribuciones siguientes ameritan un agradecimiento pero no justifican la coautoría del artículo: ayuda técnica de laboratorio, préstamo de literatura y equipo, compañía y ayuda durante viajes al campo, asistencia con la preparación de tablas e ilustraciones o figuras, sugerencias para el desarrollo de la investigación, ideas para explicar los resultados, revisión del manuscrito y apoyo económico”.

\end{thanks}
