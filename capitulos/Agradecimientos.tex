% Inicio de los Agradecimientos
\begin{thanks}

Agradezco a mis Padres por haberme dado la oportunidad.

Agradezco a la Ing. Areli Rubio Rodríguez por su inmensa ayuda en la revisión de este documento, y el apoyo con sus conocimientos.

Agradezco a mi asesor el Dr. José Fernando Estrada Saldaña por su enorme paciencia y ayuda con el desarrollo del proyecto.

[Sustituye este texto escribiendo tus agradecimientos.
La sección de agradecimientos reconoce la ayuda de personas e instituciones que aportaron significativamente al desarrollo de la investigación. No te debes exceder en los agradecimientos; agradece sólo las contribuciones realmente importantes, las menos importantes pueden agradecerse personalmente. El nombre de la agencia que financió la investigación y el número de la subvención deben incluirse en esta sección. Generalmente no se agradecen las contribuciones que son parte de una labor rutinaria o que se reciben a cambio de pago.

         Las contribuciones siguientes ameritan un agradecimiento pero no justifican la coautoría del artículo: ayuda técnica de laboratorio, préstamo de literatura y equipo, compañía y ayuda durante viajes al campo, asistencia con la preparación de tablas e ilustraciones o figuras, sugerencias para el desarrollo de la investigación, ideas para explicar los resultados, revisión del manuscrito y apoyo económico”.
]


\end{thanks}
% Fin de los Agradecimientos
