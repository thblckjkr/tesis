\section{Planteamiento del problema}

Las redes meteorológicas de alta densidad son una necesidad creciente de las áreas urbanizadas, estas redes requieren de un monitoreo y cuidado constante por personal altamente especializado.

Cuando este personal especializado no se encuentra disponible para hacer el cuidado, la calidad de los datos decrece enormemente.


Mientras más se expande una red de monitoreo meteorológico, más

Cuando una red de equipos de monitoreo meteorológicos comienza

Con la creciente importancia de los datos meteorológicos fidedignos, es importante la creación y mantenimiento de redes meteorológicas de alta densidad para el monitoreo climatológico de áreas urbanas y conurbanas.




Las redes meteorológicas de alta densidad son una necesidad creciente de la


\subsection{Antecedentes}\label{sec:Ant}

Los equipos de monitoreo climatológico funcionan de la misma forma que la mayoría de los servidores en el mercado; Un equipo de cómputo que está corriendo un programa, llamado servicio, constantemente para escuchar las peticiones de sus clientes y permitir el acceso a los datos que ha posee.

Pero de la misma forma, al ser un equipo de cómputo con funciones específicas, requiere de un alto grado de entendimiento de las funciones que realiza para poder modificarlas, así como un diagnóstico detallado y complicado para poder repararlas en caso de un fallo.

[...] Párrafo sobre los problemas de escalabilidad de las estaciones [...]. *No es posible crear una red de estaciones y mantenerlas y monitorearlas sin esfuerzo*.

[...] Párrafo puente pendiente [...]

Por lo tanto, se propone la creación de un servicio e interfaz para facilitar la administración y mantenimiento general de los equipos meteorológicos, que *has an objective to aim to a broader audience* para así reducir a los tiempos de respuesta de los fallos de las estaciones meteorológicas.

> Importancia de los datos meteoroleogicos

> Redes meteorológicas de alta densidad



Debido a la *[disponibilidad e integridad]* de los datos requerida en las estaciones meteorológicas, han buscado crear redes resilentes [...], pero eso no evita que sean completamente resistente a fallos.

Actualmente las estaciones funcionan [...].

> Nagios

\subsection{Definición del problema}

Debido a la complejidad de los sistemas de monitoreo tecnológico, y al alto grado de conocimiento que es requerido para el monitorear las estaciones y darles mantenimiento. [...] el atender los fallos de las estaciones meteorológicas lleva tiempo y expertise, aún cuando estas fallas no sean críticas o complicadas
