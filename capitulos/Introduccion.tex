% Inicio de \section*{Introducción}\label{introduccion}
\chapter*{Introducción}\label{introduccion}

%? [Redacte de manera coherente en una cuartilla cuál es la nueva contribución, su importancia y por qué es adecuado para sistemas computacionales. Se sugiere para su redacción seguir los cinco pasos siguientes: 1) Establezca el campo de investigación al que pertenece el proyecto, 2) describa los aspectos del problema que ya han sido estudiado por otros investigadores, 3) explique el área de oportunidad que pretende cubrir el proyecto propuesto, 4) describa el producto obtenido y 5) proporcione el valor positivo de proyecto.]

Las redes de monitoreo meteorológico y de calidad del aire son una necesidad creciente debido a su importancia para el monitoreo contínuo de datos y preservación para análisis, por ello, es importante crear y mantener redes densas de recolección de datos que sean estables y puedan ser mantenibles.

La baja fiabilidad de la conexión a las estaciones provocada por su acceso a sistemas de conexiones inestables, además de la instalación en lugares de difícil acceso \cite{muller_sensors_and_the_city}, provoca que sea un reto monitorearlas con sistemas convencionales. Esto, junto con la falta de un sistema homogéneo de monitoreo especializado en estaciones meteorológicas provoca que sea una tarea compleja y de alto costo el mantener una red en óptimas condiciones.

El proyecto reportado en el presente documento pretende cubrir esta área de oportunidad, desarrollando un sistema integral encargado de monitorear las estaciones meteorológicas y ejecutar acciones en las mismas, basado en la arquitectura de soluciones existentes y con un enfoque en una amplia extensibilidad. El sistema resultante es una serie de proyectos para el manejo, registro y visualización de el estado de las estaciones funcionando de manera integral para su contínuo monitoreo

%! Párrafo de cierre en el que diga como está estructurado el documento.

%! Unir los primeros dos párrafos.
