\chapter*{Introducción}\label{introduccion}

Las redes de monitoreo meteorológico y de calidad del aire son una necesidad creciente debido a su importancia para la recolección continua de datos y preservación para análisis, por ello, es importante crear y mantener redes densas de recolección de datos que sean estables y puedan ser mantenibles. La baja fiabilidad de la conexión a las estaciones provocada por su acceso a sistemas de conexiones inestables, además de la instalación en lugares de difícil acceso \cite{muller_sensors_and_the_city}, provoca que sea un reto monitorearlas con sistemas convencionales. Esto, junto con la falta de un sistema homogéneo de monitoreo especializado en estaciones meteorológicas provoca que sea una tarea compleja y de alto costo el mantener una red en óptimas condiciones.

El proyecto reportado en el presente documento pretende cubrir esta área de oportunidad, desarrollando un sistema integral encargado de monitorear las estaciones meteorológicas y ejecutar acciones en las mismas, basado en la arquitectura de soluciones existentes y con un enfoque en una amplia extensibilidad. El sistema resultante es una serie de proyectos capaces de manejar, registrar y visualizar el estado de las estaciones funcionando de manera integral para su contínuo monitoreo.

En el primer capítulo del presente documento se aborda evaluación general del estado de las tecnologías de monitoreo y sus aplicaciones en el ámbito correspondiente, posteriormente se ahonda en las definiciones necesarias para la comprensión de el mismo. En el capítulo tres, se detalla el análisis para el desarrollo del proyecto propuesto y su definición. Para terminar en una evaluación general de los resultados obtenidos y las conclusiones a las que se llegó con el desarrollo del mismo.
