% Inicio de \section*{Introducción}\label{introduccion}
\chapter*{Introducción}\label{introduccion}

%? [Redacte de manera coherente en una cuartilla cuál es la nueva contribución, su importancia y por qué es adecuado para sistemas computacionales. Se sugiere para su redacción seguir los cinco pasos siguientes: 1) Establezca el campo de investigación al que pertenece el proyecto, 2) describa los aspectos del problema que ya han sido estudiado por otros investigadores, 3) explique el área de oportunidad que pretende cubrir el proyecto propuesto, 4) describa el producto obtenido y 5) proporcione el valor positivo de proyecto.]

% Las redes de monitoreo de calidad del aire de alta densidad son una necesidad creciente de las áreas urbanizadas. Estas permiten monitorear la calidad del aire a escala local, generando información vital para la toma de desiciones en áreas de salud pública. Al incrementar la cantidad de estaciones de monitoreo, la resolución de los datos aumenta, pero también aumentan el costo, la infraestructura y el personal necesario para atenderlas \cite{urban_air_quality}. De la misma forma, la demanda para redes de monitoreo climatológico de alta densidad ha ido en aumento por su utilidad en la medición de los impactos de las políticas de control en el ambiente \cite{muller_sensors_and_the_city}.

Las redes de monitoreo meteorológico y de calidad del aire son una necesidad creciente debido a su importancia para el monitoreo continuo de datos y preservación para análisis, por ello, es el importante crear y mantener redes densas de recolección de datos que sean estables y puedan ser mantenibles.

Debido a la  \cite{muller_sensors_and_the_city}.
