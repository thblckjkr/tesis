% Inicio de \section*{Introducción}\label{introduccion}
\chapter*{Introducción}\label{introduccion}

[Redacte de manera coherente en una cuartilla cuál es la nueva contribución, su importancia y por qué es adecuado para sistemas computacionales. Se sugiere para su redacción seguir los cinco pasos siguientes: 1) Establezca el campo de investigación al que pertenece el proyecto, 2) describa los aspectos del problema que ya han sido estudiado por otros investigadores, 3) explique el área de oportunidad que pretende cubrir el proyecto propuesto, 4) describa el producto obtenido y 5) proporcione el valor positivo de proyecto.]

% Las redes de monitoreo de calidad del aire de alta densidad son una necesidad creciente de las áreas urbanizadas. Estas permiten monitorear la calidad del aire a escala local, generando información vital para la toma de desiciones en áreas de salud pública. Al incrementar la cantidad de estaciones de monitoreo, la resolución de los datos aumenta, pero también aumentan el costo, la infraestructura y el personal necesario para atenderlas \cite{urban_air_quality}. De la misma forma, la demanda para redes de monitoreo climatológico de alta densidad ha ido en aumento por su utilidad en la medición de los impactos de las políticas de control en el ambiente \cite{muller_sensors_and_the_city}.

% Las instalaciones de monitoreo climatológico eran acostumbradas a ubicarse en las afueras de complejos ubranizados, centrados en la recolección de información para un análisis a escala global de los datos climatológicos, tales como el análisis del calentamiento global y los índices de radiación, así como otros datos importantes. Debido a los cambios en las necesidades de calidad y cantidad de datos, esto ha cambiado significativamente \cite{oke_2004}, y debido a ello, las redes de monitoreo de alta densidad se han convertido en un reto logístico y económico por resolver.

% Las estaciones de monitoreo climatológico funcionan de la misma forma que la mayoría de los servidores en el mercado; Un equipo de cómputo que está corriendo un servicio escucha constantemente las peticiones de los clientes a los que está conectado, creando y actualizando datos conforme es necesario. El equipo de cómputo además se conecta a sensores que utiliza para el monitoreo contínuo de datos, los procesa, y los almacena para su posterior análisis. Esto crea la posibilidad de crear e integrar sistemas de monitoreo de equipos de cómputo para el análisis de la salud de las estaciones.

