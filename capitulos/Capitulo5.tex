\chapter{Conclusiones}\label{conclusiones}
% [Estos son los enunciados más importantes y más fuertes que se deben realizar acerca de los resultados y discusiones. Las conclusiones deben resumir el contenido y el propósito del proyecto. Se debe hacer énfasis en lo que queremos que se recuerde acerca del proyecto y realizar una síntesis de los resultados que se derivaron de los objetivos específicos trazados inicialmente (y que dieron respuesta a las preguntas de investigación si es que existen). No deben aparecer elementos nuevos o que no fueron discutidos, por ejemplo nuevos resultados observados en otros trabajos. Las conclusiones se refieren única y exclusivamente al proyecto desarrollado.

En este documento se reporta el desarrollo de un sistema de monitoreo y control para estaciones meteorológicas. A través de este apartado se presentan las conclusiones a las que se llegó durante el desarrollo del proyecto, así como las recomendaciones para trabajos futuros.

%* proyecto para titularse, decir que se aplicó lo aprendido, decir cuestiones diversas.

\section{Con respecto al objetivo de la investigación}

% Desarrollar un sistema de monitoreo y control para estaciones meteorológicas que permita un monitoreo continuo de las mismas por parte de personal no especializado, con el objetivo de proveer un mejor mantenimiento preventivo y correctivo de las estaciones meteorológicas y de calidad del aire.

%* Objetivo general entre comillas, cita textual.

Con respecto al objetivo general del proyecto, que es el desarrollo de un sistema que permita un monitoreo continuo de estaciones meteorológicas, para proveer un mejor mantenimiento preventivo y correctivo, se logró de manera exitosa y funcional en resolver el problema descrito en la sección 1.2, conforme a los resultados presentados en el capítulo 4.


Con la realización del proyecto se lograron los siguientes resultados:

%? Desarrollar un sistema central para coordinar y recolectar los datos del estado de las estaciones.

%? Diseñar un API REST para consultar el estado de las estaciones meteorológicas.

%? Diseñar e implementar una interfaz gráfica para monitorear el estatus de las estaciones.

%? Integrar las diferentes estaciones meteorológicas existentes al sistema creando los controladores correspondientes.

%? Integrar un sistema existente de notificaciones/alertas de terceros para fallos críticos de las estaciones.
\begin{itemize}
   \item Se desarrolló un sistema para recolectar los datos del estado de las estaciones, que permitió un mejor control de la información y estado de las mismas.
   %* \item Con una recolección adecuada de los datos de la sestaciones permiten un mejor control del estado de las mismas.

   \item Se diseñó un API REST para facilitar la consulta de los datos del estado de las estaciones, documentada con OpenAPI de forma automática y,

   %* EL API REST disenado facilita la consulta del estado de las estaciones. (Se puede hacer la referencia al capítulo 3)

   \item Se generó una interfaz gráfica para la consulta del estado de las estaciones, que permitió una mejor visualización de los datos, integrado con un sistema de notificación de alertas.

   %* La interfaz gráfica permite la visualización de los datos de las estaciones de una manera más visual y sencilla.

\end{itemize}



\section{Con respecto al futuro del proyecto}

Por cuestiones de que las siguientes funcionalidades se encuentran fuera del alcance de este proyecto, se recomienda que sean implementadas posteriormente:

\begin{itemize}
   \item El utilizar un sistema automatizado para la solución de incidentes de fallos en las estaciones meteorológicas.
   \item Generar reportes automatizados de calidad de datos y de líneas de tiempo de las estaciones con la información recolectada.
   \item El desarrollo de un sistema de predicción de tendencias de errores, que ayude a la toma de decisiones de mantenimiento preventivo y correctivo.
\end{itemize}

Se recomienda el automatizar



Tomar los componentes que fueron creados con tailwind y vue y realmente hacer uso de lasa clases de componentes e información.





% [La forma más simple de presentar las conclusiones es enumerándolas consecutivamente, pero podrías optar por recapitular brevemente el contenido del artículo, mencionando someramente su propósito, los métodos principales, los datos más sobresalientes y la contribución más importante de la investigación. La sección de conclusiones no debe repetir innecesariamente el contenido del resumen.]

\section{Recomendaciones para futuras investigaciones}

Debido a que las siguientes

%! Recomendaciones al laboratorio para cuando la infraestructura actual quede descontinuada.

% Fin de Conclusiones
