% Inicio de Introducción
\chapter{Conclusiones}\label{conclusiones}
[Estos son los enunciados más importantes y más fuertes que se deben realizar acerca de los resultados y discusiones. Las conclusiones deben resumir el contenido y el propósito del proyecto. Se debe hacer énfasis en lo que queremos que se recuerde acerca del proyecto y realizar una síntesis de los resultados que se derivaron de los objetivos específicos trazados inicialmente (y que dieron respuesta a las preguntas de investigación si es que existen). No deben aparecer elementos nuevos o que no fueron discutidos, por ejemplo nuevos resultados observados en otros trabajos. Las conclusiones se refieren única y exclusivamente al proyecto desarrollado.

\section{Con respecto a las preguntas de investigación}

\section{Con respecto al objetivo de la investigación}

\section{Recomendaciones para futuras investigaciones}
% [La forma más simple de presentar las conclusiones es enumerándolas consecutivamente, pero podrías optar por recapitular brevemente el contenido del artículo, mencionando someramente su propósito, los métodos principales, los datos más sobresalientes y la contribución más importante de la investigación. La sección de conclusiones no debe repetir innecesariamente el contenido del resumen.]

%! Recomendaciones al laboratorio para cuando la infraestructura actual quede descontinuada.

% Fin de Conclusiones
