\chapter{Desarrollo del Proyecto}
[Este capítulo se considera el más importante al elaborar el proyecto de titulación. Se describe el procedimiento seguido para lograr el objetivo planteado. Se explica qué y cómo se hizo, además se debe de convencer de que los métodos o procedimientos usados fueron los más adecuados.

Deben detallarse los procedimientos, técnicas, métodos, metodologías y demás estrategias metodológicas requeridas para el proyecto.
]

\section{Producto propuesto}

Se propone crear un proyecto que permita monitorear el estatus de los servicios de las estaciones meteorológicas, y por **consiguiente** de la infraestructura en la que dependen, así como ofrecer un control limitado para la solución de problemas de forma remota.

El proyecto consistirá de tres partes independientes:

\begin{itemize}
   \item Un módulo para el monitoreo de el estatus de las estaciones, que permita el cargar la información de las estaciones de una base de datos, para luego cargar los controladores específicos de cada estación basado en la infrormación existente de las mismas para después guardar el estado en el que se encuentran en una base de datos.

   \item Un API REST que tendrá el objetivo de proveer un acceso sencillo a la información de los sistemas de monitoreo, así como el de proveer una interfaz de control para las estaciones que permita la ejecución remota de comandos preestablecidos, desde cualquier punto con la autorización adecuada que haga una petición a la ruta correspondiente.

   \item Una interfaz gráfica, que permita el acceso a la información correspondiente de los sistemas de monitoreo, así como acceso a los reportes que se generen y permita capturar informes de solución de problemas de las estaciones para su posterior análisis.

\end{itemize}

\section{Fases (Metodología)}

Para el desarrollo de este programa, se utilizará la estrategia de desarrollo ágil centrada en el usuario. En ella se combina la metodología de desarrollo ágil, la cuál tiene como características principales la entrega continua de resultados y la preferencia de sistemas funcionales sobre documentaciones de código robustas \cite{agile_manifesto}, con el diseño centrado en el usuario, el cuál tiene a los usuarios como objetivo principal para satisfacer las necesidades de requerimientos.

Debido a que la experiencia de usuario es uno de los factores que pueden separar al sistema en desarrollo de los sistemas actuales de monitoreo para equipos de cómputo, el esquema de entregas, desarrollo y planeación estarán centrado en el mismo \cite{hussain_agile_usercentered}. El ciclo de entregas será con un sprint de máximo dos semanas, para una revisión de las metas, planeación y objetivos a alto nivel con el usuario y redifinir los requisitos y como sea necesario. La documentación para el usuario final, así como la documentación del API y la información técnica del sistema será un producto que será entregado al finalizar el mismo, apoyándose de la información generada en los sprints.

En el respecto del lenguaje de programación, tomando en consideración que la red de monitoreo actual utiliza WeeWX para su integración con estaciones meteorológicas \cite{red_climatologica_uacj}, así como otros componentes del sistema de monitoreo existente, se pretende utilizar Python como lenguaje principal para el desarrollo del núcleo del sistema, sus módulos, y el API de consulta. Para el desarrollo de la interfaz gráfica del sitio web, se elegirá un framework ligero con Javscript. Todo esto se empaquetará en una imagen de Docker para permitir la replicación de la instancia con el mínimo esfuerzo posible.

\section{Programa de Actividades}

{\fontfamily{lmss}\selectfont
Se pretenden llevar a cabo las actividades de acuerdo al siguiente diagrama:

\begin{table}[h]
   \centering
   % \resizebox*{!}{12 cm}{
   \begin{tabular}{|p{9cm}|c|c|c|c|c|c|c|c|c|c|}
      \hline
      ACTIVIDAD&\rotatebox{90}{Febrero 2021}
      &\rotatebox{90}{Marzo}
      &\rotatebox{90}{Abril}
      &\rotatebox{90}{Mayo}
      &\rotatebox{90}{Junio}
      &\rotatebox{90}{Julio}
      &\rotatebox{90}{Agosto}
      &\rotatebox{90}{Septiembre}
      &\rotatebox{90}{Octubre}
      &\rotatebox{90}{Noviembre 2021}\\
      \hline
      Revisión de la Literatura& \checkmark & \checkmark  & \checkmark  &  &  &  &  &  & &  \\
      \hline
      Protocolo&\checkmark &\checkmark  &\checkmark  & \checkmark &  &  &  &  & &  \\
      \hline
      Selección de herramientas & &\checkmark  & \checkmark &  &  &  &  &  & &  \\
      % \hline
      % Documentación de propuesta&  &  & \checkmark & \checkmark &\checkmark  &\checkmark  &\checkmark  &\checkmark  &\checkmark  &\checkmark  \\
      \hline
      Diseño de la interfaz de usuario &  & \checkmark &  &  \checkmark & \checkmark &  &  &  &  &  \\
      \hline
      Documentación de requerimientos &  & \checkmark & \checkmark &  \checkmark & \checkmark &  &  &  &  &  \\
      \hline
      Diseño de la base de datos&  &  &  & \checkmark & \checkmark &  &  &  &  &  \\
      \hline
      Desarrollo del núcleo del sistema&  &  &  &  & \checkmark & \checkmark &  \checkmark &  &  &  \\
      \hline
      Desarrollo de la interfaz de usuario &  &  &  &  &  & \checkmark & \checkmark &  &  &  \\
      \hline
      Desarrollo e implementación del API REST&  &  &  &  &  & \checkmark & \checkmark & \checkmark &  &  \\
      \hline
      Integración con sistema de notificaciones &  &  &  &  &  &  & \checkmark & \checkmark &  &  \\
      \hline
      Compilación y entrega de documentación&  &  &  &  &  &  &  &  & \checkmark & \checkmark \\
      \hline

      Presentación y defensa de trabajo&  &  &  &  &  &  &  &  &  & \checkmark  \\
      \hline
   \end{tabular}
	\label{Cronograma}
   \caption{Actividades a diez meses}
\end{table}
}

%! Agregar octubre para la evaluación del instrumento.

\section{Selección de base de datos}

% Casi todas las secciones de desrrollo, van ligadas a una de resultados.

Con un tiempo de respuesta de $~[N]ms$, el sistema puede soprotar hasta N estaciones concurrentes.

Debido a que la recolección de los datos es por métodología pull y no push, es posible tener las estaciones en una cola que se ejecute hasta por un periodo de 5 minutos (que es un estándar en la recolección de datos de estaciones meteorológicas). Esto implica que la base de datos [X] puede soportar hasta [N x 60 x 5] datos de forma concurrente.

Tomando en cuenta las necesidades actuales del LCCA, y el estimado del tamaño de las redes de alta densidad (que pueden llegar hasta los N nodos como X artículo lo demuestra), no vale la pena el introducir la complejidad extra de un motor de base de datos desconocido y para el que no existen ORM's con soporte completo en el lenguaje de desarrollo. Porque no es un sistema de alta densidad de datos.

Si bien es posible escalar horizontalmente la infraestructura, se busca evitarlo ya que los diminishing returns del costo de tener que mantener un sistema de monitoreo no es costeable. Para los casos de sistemas de extremadamente alta densidad, se recomienda el crear varias instancias seccionadas en bases de datos, o escalar la base de con un redis en vez de escalar.

%! Recordar que la información debe ser consultada desde el API, así que no sólo se tienen que tomar en cuenta la cantidad de query's por segundo que se requieren hacer para las inserciones, sino también para la consulta de datos.

%! Si lo que queremos es proveer herramientas para la gestión de calidad de los datos meteorológicos, la información tiene que tener en mente los principios Solidos y transaccionales, al menos en la creación de reportes basados en incidentes.

\section{Diseño de base de datos}

\section{Avances}


\section{Módulo de monitoreo}

\subsection*{Requisitos}

\subsection*{Seguridad}

\subsection*{Método de conexion}

