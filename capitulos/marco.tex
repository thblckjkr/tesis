\section{Marco Teórico}

\subsection{Marco conceptual}

\subsubsection{Estación meteorológica}

Automatic weather stations

\subsubsection{Redes meteorológicas}

\subsubsection{Red de comunicación}

Red Local, redes NAT y capas de NAT en el pais. Mencionar direcciones públicas

\subsubsection{VPN}

VPN (Incluir segmento sobre a seguridad involucrada en la VPN), y sobre la facilidad de conexion con SSH


\subsection{Marco tecnológico}

Esta es una descripción de las herramientas de tecnología que se utilizarán para el desarrollo del proyecto:

\subsubsection{Docker}

Docker es un sistema para la creación y distribución de imágenes de software, principalmente orientado a servidores, que permite el crear un ambiente replicable agnóstico al sistema operativo del host. Es un estándar en la industria de desarrollo de software para crear sistemas complejos manteniendo una relativa simpleza al desplegar nuevas instancias \cite{Rad2017DockerAnalysis}.

Además de ofrecer una mayor flexibilidad y escalabilidad para tanto para realizar pruebas en máquinas de desarrollo como para distribuir y empaquetar nuevas instancias en ambientes de producción, se ha demostrado que el costo en eficiencia al sistema que lo ejecuta es mínimo comparado con otros métodos para la administración de sistemas complejos tales como las máquinas virtuales y KVM\cite{felter2015comparsionPerformance, rad2017dockerAnalysis}.

Docker es un software y una plataforma para crear "imágenes" de software, que ayuda a tener una distribución y actualizaciones consistentes, así como el

\subsubsection{OpenAPI}

\subsubsection{FastAPI}

Modelo OSI


% SNM (Simple management network)
% OpenSource
% RaspberryPI
% Icinga
