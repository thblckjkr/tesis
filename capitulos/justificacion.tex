\section{Justificación}

Creando un sistema de monitoreo eficaz para las estaciones meteorológicas se pretende el alcanzar una mayor calidad de los datos obtenidos de las mismas, así como una mejor documentación de los sistemas meteorológicos por consecuencias. Esto pretende dar el tiempo al personal especializado en enfocarse en expandir las redes existentes meteorológicas, mejorando a largo plazo la calidad y la definición de los datos recabados con la misma cantidad de esfuerzo.

Además, haciendo el sistema de monitoreo un proyecto público y compatible con soluciones existentes, se pretende el ayudar a mejorar la calidad y confiabilidad de las redes de monitoreo meteorológico, de calidad del aire y climatológico al rededor del mundo.

\textit{**¿Mucha salsa en los tacos?**}
