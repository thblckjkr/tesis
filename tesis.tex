\documentclass[final, fmstyle, lic]{tesis}

% paquetes usados
%\usepackage[chapter, spanish]{theorems}
\newtheorem{theorem}{Teorema}[chapter]
\newtheorem{lemma}{Lema}[chapter]
\newtheorem{proposition}{Proposición}[chapter]
\newtheorem{corollary}{Corolario}[chapter]
\newtheorem{definition}{Definición}[chapter]
\newtheorem{example}{Ejemplo}[chapter]
\newtheorem{observation}{Observación}[chapter]

\usepackage{symbols}
\usepackage{graphicx} %usar imágenes externas
\usepackage{url}
\usepackage{blindtext}
\usepackage[table,xcdraw]{xcolor}

% Code highligting
\usepackage{float}
\usepackage{pdfpages}
\usepackage[chapter]{minted}

\usepackage{listings}
\usemintedstyle{friendly}
\setminted{baselinestretch=1.2}

\usepackage[utf8]{inputenc}  % descomentar para teclado en español de Unix, Mac y posiblemente Windows
\usepackage[T1]{fontenc} % this helps hyphenate Spanish vocabulary with acute accent
\usepackage{dirtytalk} % Nasty... \say
%\usepackage[latin1]{inputenc}  % descomentar para teclado en español de Unix y posiblemente también de Windows
%\usepackage[applemac]{inputenc}  % descomentar para teclado en español de Mac
\usepackage[spanish]{babel}
\decimalpoint

\setlength{\parskip}{2mm}	%Espaciado entre párrafos

\usepackage{hyperref}
%\usepackage{urm}
% datos de la tesis
\title{SISTEMA DE MONITOREO Y CONTROL PARA ESTACIONES METEOROLÓGICAS}
\author{142912}{Teo González Calzada}
\authorb{}{}
\director{Mtro.}{José Fernando Estrada Saldaña}
\directorb{Dr.}{Felipe Adrián Vazquez Galvez}
\directorbp{«Profesor-Investigador IIT»}  % Dejar en campo en blanco si no hay segundo asesor
\sinodala{Dr.}{Nombre del Sinodal A}
\sinodalb{Dr.}{Nombre del Sinodal B}
\sinodalc{Dr.}{Nombre del Sinodal C}
\titular{Dr.}{Everardo Santiago Ramírez} % Titular de Seminario de Titulación II

%? fecha en la primera página

% comenta para automático:
\termyear{\today}

% Permite que los listings con minted y con lstlisting tengan los mismos números
% https://tex.stackexchange.com/questions/292569/listings-internationalization-spanish-list-of-listings-and-caption-labels-not
\makeatletter
\AtBeginDocument{\let\c@lstlisting\c@listing}
\makeatother

\begin{document}
\pagenumbering{roman}
\renewcommand{\lstlistingname}{Listado}
\renewcommand{\listingscaption}{Listado}
\renewcommand\listoflistingscaption{Lista de listados}

\maketitle     % esto hace las portadas
\approvalpage  % esto hace la pagina de aprobaci√õn
\approvalprint  % esto hace la pagina de aprobaci√õn
%\approvalprint % esto hace la pagina de aprobaci√õn de impresión. Comentar si sólo un alumno
\originalpage

% Inicio de los Agradecimientos
\begin{thanks}

[Sustituye este texto escribiendo tus agradecimientos.
La sección de agradecimientos reconoce la ayuda de personas e instituciones que aportaron significativamente al desarrollo de la investigación. No te debes exceder en los agradecimientos; agradece sólo las contribuciones realmente importantes, las menos importantes pueden agradecerse personalmente. El nombre de la agencia que financió la investigación y el número de la subvención deben incluirse en esta sección. Generalmente no se agradecen las contribuciones que son parte de una labor rutinaria o que se reciben a cambio de pago. 

         Las contribuciones siguientes ameritan un agradecimiento pero no justifican la coautoría del artículo: ayuda técnica de laboratorio, préstamo de literatura y equipo, compañía y ayuda durante viajes al campo, asistencia con la preparación de tablas e ilustraciones o figuras, sugerencias para el desarrollo de la investigación, ideas para explicar los resultados, revisión del manuscrito y apoyo económico”.
]


\end{thanks}
% Fin de los Agradecimientos
\setcounter{page}{6}
% Inicio de Dedicatoria

\chapter*{Dedicatoria}

A mi padre, que ya no está conmigo, y a mi madre y hermana que lo están, por su apoyo incondicional durante todos estos años.


% Inicio de Resumen
\chapter*{Resumen}
\markboth{chapter name}{Sinopsis}


[Sustituye este texto escribiendo tu sinopsis o resumen. Es un panorama general de todo lo que el lector encontrará en tu documento, en no más de una página. Recuerda que este, junto con el título, son la parte más leída de tu documento cuando alguien más lo busca en las bases de datos, el “punto de venta''.]


% Fin del Resumen

% los siguientes comandos producen Ìndices.
% tablas y figuras en el documento de la tesis
\tableofcontents
\listoffigures
% \listoftables
\listoflistings

\setcounter{page}{1}

\setcounter{page}{1}
\pagenumbering{arabic}
% Inicio de \section*{Introducción}\label{introduccion}
\chapter*{Introducción}\label{introduccion}

%? [Redacte de manera coherente en una cuartilla cuál es la nueva contribución, su importancia y por qué es adecuado para sistemas computacionales. Se sugiere para su redacción seguir los cinco pasos siguientes: 1) Establezca el campo de investigación al que pertenece el proyecto, 2) describa los aspectos del problema que ya han sido estudiado por otros investigadores, 3) explique el área de oportunidad que pretende cubrir el proyecto propuesto, 4) describa el producto obtenido y 5) proporcione el valor positivo de proyecto.]

% Las redes de monitoreo de calidad del aire de alta densidad son una necesidad creciente de las áreas urbanizadas. Estas permiten monitorear la calidad del aire a escala local, generando información vital para la toma de desiciones en áreas de salud pública. Al incrementar la cantidad de estaciones de monitoreo, la resolución de los datos aumenta, pero también aumentan el costo, la infraestructura y el personal necesario para atenderlas \cite{urban_air_quality}. De la misma forma, la demanda para redes de monitoreo climatológico de alta densidad ha ido en aumento por su utilidad en la medición de los impactos de las políticas de control en el ambiente \cite{muller_sensors_and_the_city}.

Las redes de monitoreo meteorológico y de calidad del aire son una necesidad creciente debido a su importancia para el monitoreo continuo de datos y preservación para análisis, por ello, es el importante crear y mantener redes densas de recolección de datos que sean estables y puedan ser mantenibles.

Debido a la  \cite{muller_sensors_and_the_city}.

\chapter{Planteamiento del Problema}

\section{Antecedentes}

El desplegar y mantener una red meteorológica urbana compone bastantes retos: Entre la creciente dificultad de crear sistemas de medición estandarizados que se adapten al siempre cambiante paisaje urbano; como la instalación de los equipos de medición y de guardado de datos en áreas que permitan acceso para mantenimiento y que sean seguros; y la dificultad de encontrar un punto de acceso a internet adecuado para transferir la información generada, el generar una red de monitoreo es una tarea extensa y compleja.

Debido a estos retos, la comunidad de monitoreo climatológico y meteorológico se ha enfocado en la creación de sistemas que sean más eficientes y económicos. Entre estos esfuerzos, se encuentra el amplio uso de RaspberryPi como centro de recolección de datos de estaciones de monitoreo \cite{rpi_weataher_station}, tanto caseras como profesionales, con la ayuda de sistemas abiertos para la recolección de datos como lo es WeeWX. Esto ha hecho factible el \textit{levantar} redes de 50 nodos de monitoreo con sensores económicamente viables para actores con un presupuesto limitado \cite{monitoreo_raspberry_nagios}.

% TODO Levantar, en el parrafito anterior no me agrada por completo, pero lo dejo debido a la falta de una mejor palabra.

% TODO! Agregar sección de monitoreo de Campbell, y como funciona actualmente

Estas redes densas requieren de un monitoreo contínuo para mantener una alta calidad de los datos recabados, y evitar las pérdidas por falta de mantenimiento. Entre los sistemas de monitoreo que pueden ser adaptados para el monitoreo de estaciones meteorológicas y la red que las soporta, se encuentra la plataforma Nagios®, el cual es un sistema de monitoreo contínuo orientado a redes y servidores. Entre la información que recaba Nagios contínuamente para el estado de los servidores, se encuentra el uso de CPU y RAM, así como estado de los discos, puertos, e información variada de servicios de red en los hosts. Debido a que Nagios es un sistema de monitoreo de redes orientado a profesionales de la informática, la interfaz gráfica es poco amigable con los usuarios menos familiarizados con los conceptos técnicos de los sistemas computacionales \ref{fig:nagios_dashboard}.

\begin{figure}[!ht]
	\centering
	\includegraphics[width=.80\linewidth]{images/Nagios_home_dashboard.png}
	\caption{Tablero principal de Nagios XI.}
	\label{fig:nagios_dashboard}
\end{figure}

Si bien Nagios ofrece la posibilidad de monitorear parámetros adicionales con un sistema establecido de plugins en python y otros lenguajes, además de poseer una fuerte comunidad que crea contínuamente plugins relacionados con el proyecto, hasta el momento la cantidad de plugins relacionados con estaciones meteorológicas existentes es mínima ya que los esfuerzos de la comunidad se centran principalmente en el monitoreo de centros de datos, redes y routers.

Además de los plugins existentes en la comunidad de Nagios, existe la alternativa abierta conocida como \textit{monitoring plugins} \cite{monitoring_plugins}, que es una plataforma compatible con diversos sistemas de monitoreo de redes y servicios, en la cuál es posible encontrar una mayor cantidad de scripts de montioreo relacionados con los sistemas de recolección de datos meteorológicos. La utilización de estos plugins abiertos junto con un sistema central como Nagios ya ha sido propuesto anteriormente \cite{monitoreo_raspberry_nagios}.

Las limitaciones de Nagios vienen a que debido a su implementación orientada a sistemas de alta disponibilidad, la configuración de los parámetros de tolerancia para la resilencia a fallas son generalmente limitados en la variedad de los mismos y los valores fijos que se pueden establecer. Además, debido a las necesidades de seguridad y accesibilidad de las estaciones meteorológicas en el contexto urbano, estas suelen instalarse en sistemas en los que se posee poco control de la red que les provee comunicación, como son las escuelas, hospitales y estaciones de policía, así como otros espacios públicos \cite{muller_sensors_and_the_city}, dificultando aún más la capacidad de monitoreo y disponibilidad de la red y los sistemas que soporta. Esta limitante se hace más presente debido a que los sistemas se vuelven completamente dependientes de una VPN para funcionar y para su control, debido al extenso uso de redes NAT IPv4 que predominan en los sitios de instalación.

Otra de las limitantes del sistema Nagios es que al ser un sistema de monitoreo limita la capacidad de acción de los usuarios ante un caso que lo requiera. Si bien es posible el realizar acciones por medio de scripts creados al momento de la configuración de Nagios, estos requieren una configuración extensa y compleja \cite{nagios_service_restart}. Además, el sistema basado en eventos impide la interacción directa de un usuario para la respuesta de forma directa desde la consola de administración, requiriendo que el usuario tenga un conocimiento del método de conexión así como la información necesaria para realizar una tarea trivial como es el reiniciar un servicio.

% Pero de la misma forma, al ser un equipo de cómputo con funciones específicas, requiere de un alto grado de entendimiento de las funciones que realiza para poder modificarlas, así como un diagnóstico detallado y complicado para poder repararlas en caso de un fallo.

% Debido a la *[disponibilidad e integridad]* de los datos requerida en las estaciones meteorológicas, han buscado crear redes resilentes [...], pero eso no evita que sean completamente resistente a fallos.

Actualmente, el sistema de monitoreo de calidad del aire y climatológico de la Universidad Autónoma de Ciudad Juárez, engloba varios de los retos antes descritos, ya que a pesar de la baja densidad de estaciones meteorológicas, comprende una variedad importante de las mismas. Actualmente, se compone por prototipos conectados por medio de puertos expuestos por NAT monitoreados remotamente \cite{red_climatologica_uacj}, estaciones de diferentes proveedores siendo monitoreadas local y externamente, así como estaciones remotas con routers dentro de la red local de la universidad \ref{fig:current_network}. Lo que provoca que sea un reto el monitorearla adecuadamente debido a la variedad de sus componentes.

\begin{figure}[!ht]
	\centering
	\includegraphics[width=.80\linewidth]{images/diagrams/current_network.png}
	\caption{Diagrama de red de LCCA UACJ.}
	\label{fig:current_network}
\end{figure}

Una parte vital de los sistemas de monitoreo es el registro detallado de incidentes para su posterior análisis o referencia y actualmente, no se cuenta con un registro de causas y soluciones, una base de conocimientos para los fallos de las estaciones meteorológicas existentes, así como tampoco existe un reporte automático de la calidad de los datos meteorológicos derivado de los diagnósticos y estado de salud de las estaciones meteorológicas, lo cual dificulta el uso de la información recabada por las mismas por los usuarios qeu accedan a la información que es recabada.


\section{Definición del problema}

La falta de una plataforma estandarizada para el monitoreo de estaciones meteorológicas que permita un monitoreo contínuo y control de diversos tipos de estaciones, así como la disponibilidad del acceso a los datos, y un registro de incidentes es una dificultad creciente. Con redes cada vez más densas que son cada vez más accesibles económicamente y menos complejas de crear, se hace notar la necesidad de un sistema estandarizado y compatible con soluciones existentes para el monitoreo de las estaciones.

%! TODO: Posibilidad de integrar con FarmOS y otros. Enfocarse en la amplia disponibilidad y accesibilidad general.

%! Tiempo de respuesta como aporte secundario en la metodología


\section{objetivo general}

Desarrollar un sistema de monitoreo y control para estaciones meteorológicas que permita un monitoreo continuo del personal no especializado al cuidado de las mismas, con el objetivo de proveer un mejor mantenimiento preventivo y correctivo de las estaciones para así obtener una mejor calidad de datos meteorológicos.

\subsection{Objetivos específicos}

\begin{itemize}
   % \item Hacer un sistema modular y extendible para el monitoreo de las estaciones meteorológicas.

   \item Desarrollar un sistema central para coordinar y recolectar los datos de las estaciones.

   \item Diseñar un API REST para consultar estatus de las estaciones meteorológicas.

   \item Diseñar e implementar una interfaz gráfica para monitorear el estatus de las estaciones.

   \item Integrar las diferentes estaciones meteorológicas existentes al sistema creando los controladores correspondientes.

   \item Integrar un sistema existente de notificaciones/alertas de terceros para fallos críticos de las estaciones.
\end{itemize}

\section{justificación}

Creando un sistema de monitoreo eficaz para las estaciones meteorológicas se pretende el alcanzar una mayor calidad de los datos obtenidos de las mismas, así como una mejor documentación de los sistemas meteorológicos por consecuencias. Esto pretende dar el tiempo al personal especializado en enfocarse en expandir las redes existentes meteorológicas, mejorando a largo plazo la calidad y la definición de los datos recabados con la misma cantidad de esfuerzo.

Además, haciendo el sistema de monitoreo un proyecto público y compatible con soluciones existentes, se pretende el ayudar a mejorar la calidad y confiabilidad de las redes de monitoreo meteorológico, de calidad del aire y climatológico al rededor del mundo.

\subsection{Alcances y limitaciones}

Si bien se pretende que el proyecto tenga la flexibilidad suficiente para adaptarse a nuevos casos de uso sin necesidad de reescribir grandes partes del mismo, también se consideran varios límites:

La integración de estaciones meteorológicas se limitará a cubrir casos conocidos y recurrentes de estaciones meteorológicas existentes, como lo son las estaciones Davis Vantage Pro® conectadas a un RaspberryPi como DataLogger, así como las estaciones Campbell® accesadas directamente por su dirección IP y a las que se accesa por medio de una RaspberryPi que funciona como DataLogger. El proyecto estará limitado a conectarse a solamente una de cada uno de los tipos de estaciones meteorológicas existentes.

Así mismo, la interfaz gráfica, y el diseño y la implementación del API se limitarán a cubrir los casos comunes de fallo de las estaciones anteriormente mencionadas. Respecto a los servicios que el sistema monitoreará, se pretende que sólo se monitoreen los escenciales para el monitoreo y funcionamiento correcto del sistema de recolección de datos existentes, tales como el servicio WeeWX, el proveedor de VPN y el servicio de backups por estación si es que existe. De la misma forma, el proyecto se limitará a adaptarse a la infraestructura de red existente y a integrar estaciones que se encuentren recolectando datos.

\chapter{Marco referencial}

%! Párrafo introductorio

\section{Marco teórico}

% [Es la selección, exposición y análisis de la o las teorías, métodos, procedimientos y conocimientos que sirven para fundamentar el tema, explicar los antecedentes e interpretar los resultados de la investigación. La teoría constituye la base donde se sustentará cualquier análisis, experimento o propuesta de desarrollo de un trabajo de grado.]

% Aqui se presenta la información general, en el desarrollo se pone cómo se usó la información.

% \subsection{Estación meteorológica}


% Automatic weather stations

% %! Agregar un tema general

% \subsection{Redes meteorológicas}

% \subsection{Redes de comunicación}

% Red Local, redes NAT y capas de NAT en el pais. Mencionar direcciones públicas

% \subsection{Modelo OSI}

% \subsection{VPN}

% Seguridad de la información

% Protocolo SSH y (Network monitoring)

% VPN (Incluir segmento sobre a seguridad involucrada en la VPN), y sobre la facilidad de conexion con SSH


% \subsection{API REST}

Los sistemas informáticos que se componen de más de un componente, utilizan diversos métodos de comunicación entre ellos. Desde el accesar directamente a localizaciones de memoria física o virtual en un dispositivo para compartir información hasta crear librerías compartidas entre sistemas para accesar a la información en un depósito externo (conocidas como APIs), cada forma de acceso a los datos tiene su propio nivel de abstracción, oportunidades, y desventajas, las cuales deben ser evaluadas antes de elegir una tecnología adecuada para responder a las necesidades de cada proyecto.

%! Bibliotecas de software, no bibliotecas
%! ¿Cambiar el término de API por interfaces? (A menos que las interfaces estén en bibliotecas).

Un API REST es un estándar de acceso a la información de sistemas externos por medio de protocolos \textit{Web}, tales como HTTP/HTTPS, los cuáles permiten la consulta de datos en cualquier lenguaje que permita realizar conexiones y consultas a sitios web \cite{REST_API_design}. Entre las principales características de los API REST se encuentra que no es necesario proveer un estado previo para acceder a la información, esto implica que no es necesario mas que conocer la ruta en la que se encuentran los datos requeridos para acceder a ellos. Las ventajas que ofrece es la amplia disponibilidad de acceso a los datos y la fácil integración con sistemas existentes de manejo y procesamiento de información \cite{OpenAPI_example}.

%! Mencionar esta figura!!

\begin{figure}[!ht]
	\centering
	\includegraphics[width=.70\linewidth]{images/diagrams/REST.png}
	\caption{Diagrama del protocolo REST.}
	\label{fig:coms_nodos_raspberry}
\end{figure}

% \subsection{OpenAPI}

Debido a la naturaleza libre de el desarrollo web, y la poca estandarización de la comunicación entre los clientes web con los servidores, se creó la iniciativa OpenAPI a partir de un estándar existente proveído por la compañía Smartbear, Swagger. Este estándar para la comunicación con sitios por medio de APIs REST rápidamente fué ganando popularidad gracias a la fundación Linux hasta convertirse en un estándar utilizado ampliamente por diversas organizaciones y empresas \cite{OpenAPI_foundation}.

El estándar OpenAPI es un esquema de definición de estructura de datos en JSON. En este esquema se definen las rutas a las cuáles se puede acceder, los parámetros que aceptan y sus respectivos tipos de datos, así como la información que responde y los tipos de datos de los mismos. Y debido a la naturaleza abierta del esquema, este puede ser generado e integrado nativamente con el uso de diversas tecnologías de desarrollo. Permitiendo, por ejemplo, el generar las clases e interfaces correspondientes para el uso por medio de clases de los datos para lenguajes de programación fuertemente tipados \cite{openapi_generator}.

\section{Marco tecnológico}

%! Desarrollar el marco tecnológico.

%! Aplicaciones, capacidades, performance, etc

A continuación se presenta una descripción de las herramientas de tecnología que se utilizarán para el desarrollo del proyecto:

\subsection{Docker}

Docker es un sistema para la creación y distribución de imágenes de software, principalmente orientado a servidores, que permite el crear un ambiente replicable agnóstico al sistema operativo del host. Es un estándar en la industria de desarrollo de software para crear sistemas complejos manteniendo una relativa simpleza al desplegar nuevas instancias \cite{rad2017dockerAnalysis}.

%!  Palabra figura

Docker utiliza un sólo Kérnel de linux para la creación de los contenedores y cada uno de los contenedores puede contener hasta \textit{n} procesos, lo que lo ayuda a reducir el tamaño de sistemas complejos \ref{fig:docker_diagrama}. Además de ofrecer una mayor flexibilidad y escalabilidad para tanto para realizar pruebas en máquinas de desarrollo como para distribuir y empaquetar nuevas instancias en ambientes de producción, se ha demostrado que el costo en eficiencia al sistema que lo ejecuta es mínimo comparado con otros métodos para la administración de sistemas complejos tales como las máquinas virtuales y el empaquetado en KVM \cite{rad2017dockerAnalysis, felter2015comparsionPerformance}.


\begin{figure}[!ht]
	\centering
	\includegraphics[width=.45\linewidth]{images/diagrams/docker.png}
	\caption{Diagrama del contenedor de procesos Docker.}
	\label{fig:docker_diagrama}
\end{figure}

\subsection{Masonite ORM}

Masonite ORM es una solución creada para python que permite la manipulación de sistemas relacionales de bases de datos creando una interfaz de código. Abstrae la complejidad de la manipulación de base de datos para convertirla en un modelo de clases con una interfaz simple para la edición de los datos. Tiene soporte nativo para transacciones, es compatible con MariaDB y está diseñado para ser incluido en proyectos complejos sin necesidad de incluir un framework completo \cite{masonite_2021}.

\subsection{FastAPI}

FastAPI es un framework para desarrollo de APIs REST para Python centrado en el desarrollo rápido con ayuda de las anotaciones estándar de Python. Además de ser uno de los frameworks de desarrollo más rápidos en su ejecución, permite el crear directamente documentación compatible con el estándar OpenAPI sin necesidad de librerías externas \cite{fastapi_ramirez_2020}. Todo esto lo convierte en un framework ideal para extender proyectos existentes en Python y con su permisiva licencia permite % TODO terminar esta oración

\subsection{MariaDB}

%! Quitar esto de aquí, cambiar a la selección de base de datos.

MariaDB es un motor de base de datos relacional creado por el equipo original que desarrolló MySQL, es un motor que tiene como objetivo mantenerse completamente abierto y tiene una licencia de uso permisiva para su uso en ambientes comerciales y no comerciales \cite{mariadb_foundation_2019}. Tiene un rendimiento similar a MySQL en operaciones transaccionales, por lo cual es una excelente alternativa cuando se requiere un modelo de licencias permisivo \cite{mariadb_comparison}.

%  TODO: Consider the use of a time series database.

% TODO \subsection{React} Sección de tecnología para front, probablemente React

% SNM (Simple management network)
% OpenSource
% RaspberryPI
% Icinga


\chapter{Desarrollo del Proyecto}
% [Este capítulo se considera el más importante al elaborar el proyecto de titulación. Se describe el procedimiento seguido para lograr el objetivo planteado. Se explica qué y cómo se hizo, además se debe de convencer de que los métodos o procedimientos usados fueron los más adecuados.

% Deben detallarse los procedimientos, técnicas, métodos, metodologías y demás estrategias metodológicas requeridas para el proyecto.
% ]

En este capítulo se detallará el proceso de diseño e implementación que se realizó para desarrollar el proyecto de monitoreo de estaciones meteorológicas, así como las limitaciones técnicas para el desarrollo de

%! Punto al final de todos los títulos de las imágenes

\section{Producto propuesto}

Se creó un proyecto que permita monitorear el estado de los servicios de las estaciones meteorológicas, así como de la infraestructura en la que dependen, así como ofrecer un control limitado para la solución de problemas de forma remota.


\noindent{El proyecto consiste de tres partes independientes:}

\begin{itemize}
   \item Un módulo para el monitoreo de el \textit{estatus} de las estaciones, que permita el cargar la información de las estaciones de una base de datos, para luego cargar los controladores específicos de cada estación basado en la infrormación existente de las mismas para después guardar el estado en el que se encuentran en una base de datos.

   \item Un API REST que tendrá el objetivo de proveer un acceso sencillo a la información de los sistemas de monitoreo, así como el de proveer una interfaz de control para las estaciones que permita la ejecución remota de comandos preestablecidos, desde cualquier punto con la autorización adecuada que haga una petición a la ruta correspondiente.

   \item Una interfaz gráfica, que permita el acceso a la información correspondiente de los sistemas de monitoreo, así como acceso a los reportes que se generen y permita capturar informes de solución de problemas de las estaciones para su posterior análisis.

\end{itemize}

\section{Metodología de desarrollo}

%? Usar plantilla correcta para el capítulo 3 (Se debe hablar en pasado en capítulo 3)

Para el desarrollo de este programa, se utilizó la estrategia de desarrollo ágil centrada en el usuario. En ella se combina la metodología de desarrollo ágil, la cuál tiene como características principales la entrega continua de resultados y la preferencia de sistemas funcionales sobre documentaciones de código robustas \cite{agile_manifesto}, con el diseño centrado en el usuario, el cuál tiene a los usuarios como objetivo principal para satisfacer las necesidades de requerimientos.

%-? Imagen para ilustrar la metodología de desarrollo ágil
%! Cambiar esta imagen al español
\begin{figure}[!ht]
	\centering
	\includegraphics[width=.75\linewidth]{images/agil.png}
	\caption{Diagrama de metodología ágil.}
	\label{fig:conexion_redundancia}
\end{figure}

Debido a que la experiencia de usuario es uno de los factores que pueden separar al sistema en desarrollo de los sistemas actuales de monitoreo para equipos de cómputo, el esquema de entregas, desarrollo y planeación estarán centrado en el mismo \cite{hussain_agile_usercentered}. El ciclo de entregas será con un sprint de máximo dos semanas, para una revisión de las metas, planeación y objetivos a alto nivel con el usuario y redifinir los requisitos y como sea necesario. La documentación para el usuario final, así como la documentación del API y la información técnica del sistema será un producto que será entregado al finalizar el mismo, apoyándose de la información generada en los sprints.

%? X que se reporta, fue desarrollando utilizando la metodología X y consta de X fase y se realizó X cosa en cada fase. Cada una se explica de forma general.

%? Quitar el siguiente texto, no corresponde a la metodología.

En el respecto del lenguaje de programación, tomando en consideración que la red de monitoreo actual utiliza WeeWX para su integración con estaciones meteorológicas \cite{red_climatologica_uacj}, así como otros componentes del sistema de monitoreo existente, se pretende utilizar Python como lenguaje principal para el desarrollo del núcleo del sistema, sus módulos, y el API de consulta. Para el desarrollo de la interfaz gráfica del sitio web, se elegirá un framework ligero con Javscript. Todo esto se empaquetará en una imagen de Docker para permitir la replicación de la instancia con el mínimo esfuerzo posible.

% \subsection{Programa de Actividades}
%-? Esto está relacionado con la 3.2, debe ser una ssección de la 3.2

%-? Esto es una tabla, no un diagrama. En la tabla X se muestra lo siguiente.
Este proyecto se desarrolló de acuerdo a las actividades que se muestran en la Tabla \ref{Cronograma}.

%! Debido a X y Y, las fases Z se hicieron en X fecha.

{\fontfamily{lmss}\selectfont


%-? Las tablas deben tener descripción en la parte superior, y debe tener un punto final

\begin{table}[H]
   \centering
   \caption{Actividades a diez meses.}
   \label{Cronograma}
   % \resizebox*{!}{12 cm}{
   \begin{tabular}{|p{8cm}|c|c|c|c|c|c|c|c|c|c|}
      \hline
      ACTIVIDAD&\rotatebox{90}{Febrero 2021}
      &\rotatebox{90}{Marzo}
      &\rotatebox{90}{Abril}
      &\rotatebox{90}{Mayo}
      &\rotatebox{90}{Junio}
      &\rotatebox{90}{Julio}
      &\rotatebox{90}{Agosto}
      &\rotatebox{90}{Septiembre}
      &\rotatebox{90}{Octubre}
      &\rotatebox{90}{Noviembre 2021}\\
      \hline
      Revisión de la Literatura& \checkmark & \checkmark  & \checkmark  &  &  &  &  &  & &  \\
      \hline
      Protocolo&\checkmark &\checkmark  &\checkmark  & \checkmark &  &  &  &  & &  \\
      \hline
      Selección de herramientas & &\checkmark  & \checkmark &  &  &  &  &  & &  \\
      % \hline
      % Documentación de propuesta&  &  & \checkmark & \checkmark &\checkmark  &\checkmark  &\checkmark  &\checkmark  &\checkmark  &\checkmark  \\
      \hline
      Diseño de la interfaz de usuario &  & \checkmark &  &  \checkmark & \checkmark &  &  &  &  &  \\
      \hline
      Documentación de requerimientos &  & \checkmark & \checkmark &  \checkmark & \checkmark &  &  &  &  &  \\
      \hline
      Diseño de la base de datos&  &  &  & \checkmark & \checkmark &  &  &  &  &  \\
      \hline
      Desarrollo del núcleo del sistema&  &  &  &  & \checkmark & \checkmark &  \checkmark &  &  &  \\
      \hline
      Desarrollo de la interfaz de usuario &  &  &  &  &  & \checkmark & \checkmark &  &  &  \\
      \hline
      Desarrollo e implementación del API REST&  &  &  &  &  & \checkmark & \checkmark & \checkmark &  &  \\
      \hline
      Integración con sistema de notificaciones &  &  &  &  &  &  & \checkmark & \checkmark &  &  \\
      \hline
      Compilación y entrega de documentación&  &  &  &  &  &  &  &  & \checkmark & \checkmark \\
      \hline

      Presentación y defensa de trabajo&  &  &  &  &  &  &  &  &  & \checkmark  \\
      \hline
   \end{tabular}
\end{table}
}

\clearpage

\section{Análisis y especificación de requisitos}

Debido a la naturaleza autónoma de las estaciones meteorológicas, y a que el hecho que las mismas se encuentran sometidas a [something something] se busca crear un sistema centralizado de recolección de información que tenga gran tolerancia a las diversas condiciones adversas que se enfrentan las estaciones meteorológicas, a la vez que es lo suficientemente confiable para hacer un impacto positivo en la recolección de la información de las mismas.

\subsection{Conexión a estaciones remotas}

La conexión a las estaciones remotas se creó como un sistema modular de conexiones. Teniendo el objetivo de la extensibilidad como objetivo prioritario para el sistema de interacción con las interfaces.

Cada sistema de conexión supone sus propios retos, si bien hay diversos métodos de conexión que podrían ser útiles para la conexión a las estaciones meteorológicas, se decidió enfocarse en la conexión vía SSH a las estaciones meteorológicas que poseen una RaspberryPI como *datalogger* y como medio de interfaz que se encuentran conectadas por medio de puerto serial a las mismas. Y de las estaciones meteorológicas Campbell, que poseen diversos protocolos de comunicación pero se decidió por utilizar el protocolo HTTP.

Para la conexión a las estaciones RaspberryPi se considera lo siguiente:

\begin{itemize}
   \item Actualmente cuentan con una VPN configurada para facilitar el acceso a SSH por medio de una dirección IP en el mismo segmento de red que el segmento al que se pretende el servidor final tenga.
   \item Ocasionalmente, las estaciones meteorológicas perderán acceso a la VPN, ya sea por fallas técnicas del servidor, del ISP, pérdidas de energía eléctrica o demás.
   \item Que una estación se encuentre fuera de línea de la VPN temporalmente no implica que esta no pueda operar, o incluso que no pueda contactar al servidor, tal como se observa en la \ref{fig:conexion_redundancia}
\end{itemize}

\begin{figure}[!ht]
	\centering
	\includegraphics[width=.75\linewidth]{images/diagrams/conexion.png}
	\caption{Diagrama de la redundancia de las conexiones}
	\label{fig:conexion_redundancia}
\end{figure}

Por esta razón se optó por tener un servicio de monitoreo bidireccional. Se pretende que cambiando el ejecutor de servicios, se pueda obtener la información de la estación meteorológica sin necesidad de realizar diferentes implementaciones para cada caso. En este caso, se pretende que un script funcione en el mismo

\subsubsection{Consideraciones de seguridad}

Debido a que generalmente no se crea una red virtual privada separada para el manejo exclusivo de estaciones meteorológicas (ya que estas suelen instalarse sobre infraestructura existente) es importante tener consideraciones de seguridad respecto a el acceso a las estaciones, debido a que pueden ser un punto de acceso a una, otherwise, isolada y segura red.


\textbf{De la conexión del servidor a las estaciones meteorológicas}

Para realizar la conexión a las estaciones meteorológicas se requiere de acceso a la raspberrypi que funciona como puente entre ambas. Para realizar cambios, crear un servicio, y establecer la información del sistema con una mínima interacción se requiere de un usuario de alta prioridad a la máquina. En el caso del sistema operativo basado en linux que utilizan las estaciones, es el usuario con la mayor cantidad de procesos \emph{root}.

Al considerarse comprometido el ambiente de la apliación, se consideraría comprometido el sistema completo. Ya que en este ambiente se encontrarán las contraseñas de acceso a la base de datos y la llave privada que se utiliza para hacer autenticación, si bien existen servicios como Aws-KMS (Key Management Service), el implementar un sistema tan robusto para la administración de secretos sale de los objetivos de este proyecto.

Por lo tanto, se decidió crear un servicio que tome un usuario y password con acceso "root" de forma temporal (o al menos uno que tenga permisos de \emph{sudoer}) y utilizarlo para almacenar la llave pública local del servidor para realizar operaciones sin tener un usuario/password almacenado en la base de datos que pudiera ser comprometido. De esta forma, se mitiga el impacto de una posible intrusión a la base de datos, para no comprometer las credenciales de acceso a las estaciones.

\textbf{De las estaciones meteorológicas al servidor}

Debido a que las estaciones meteorológicas suelen ser instaladas en puntos con poco o mínimo control de seguridad física, se busca mitigar el acceso de las estaciones meteorológicas a la base de datos en la que se centralizarán los datos. Por lo tanto, se decidió utilizar un protocolo de API para insertar los eventos.


\subsection{Selección de herramientas}

Para el desarrollo de el proyecto, se decidió por realizar la aplicación de servidor con el lenguaje de programación Python. Esto debido a que otros proyectos de los que depende el funcionamiento de los sistemas de el LCCA, tales como el monitoreo climatológico y meteorológico con Weewx, y el proyecto para obtener la información de las estaciones meteorológicas en diferentes estándares fueron realizados con este lenguaje. Además, el lenguaje cuenta con una librería estándar extensa así como una librería de terceras partes madura que permite el desarrollo de forma sencilla utilizando estas librerías existentes, contando con la certeza de que están listas para un proyecto de producción.

%! TODO: Agregar referencias a las tesis de RapsberryPI y la de Alicia en el parrafito anterior.

El \textit{ORM} utilizado para el desarrollo de la aplicación, fué \textit{MasoniteORM}, un ORM para python que tiene como objetivos principales la simpleza y extendibilidad de proyectos. Si bien, \textit{MasoniteORM} es parte de \textit{Masonite}, un framework para el desarrollo de aplicaciones web, este es bastante extenso y complejo, y si bien es fácilmente extensible no es simple para usarse. Por lo tanto, se decidió utilizar el framework de desarrollo de aplicaciones web \textit{FastApi}, creado por Sebastián Ramírez (tiangolo), ya que ofrece una forma fácil de crear un \textit{API} web, el cuál será utilizado posteriormente para el desarrollo de una interfaz fácilmente accesible para los usuarios.

La interfaz gráfica para proveer acceso a la información a los usuarios se decidió hacer en una aplicación web. Esto debido a que las interfaces web ofrecen una amplia y madura plataforma desarrollo que se puede acceder desde diferentes tipos de dispositivos, así como una gran variedad de \textit{frameworks}, metodologías, y paradigmas, lo que ofrece una gran flexibilidad al momento de realizar un desarrollo a la medida. También está el hecho de que debido a la naturaleza del proyecto, un sistema que centraliza toda la información que es accesible por medio de un API, no parecía posible que un proyecto de una aplicación de escritorio o una aplicación web ofreciera una ventaja que no ofreciera la interfaz web.

VueJS
%! TODO: Agregar descripción de VueJS y las referencias adecuadas a la tesis.

Por el motivo de hacer el proyecto lo más estándar, fácilmente acccesible y mantenible, se decidió realizar la programación y documentación del proyecto en inglés, pero manteniendo las interfaces en las que el usuario interactúa con el mismo en español. Además, se eligió el configurar \textit{pylint} con el estándar \textit{pep8} para el formato automático de el código del proyecto en el estándar. De la misma forma, se configuró \textit{ESLint} en el proyecto de VueJS para el fronted, extendiendo los estándares \textit{vue:essential} y \textit{eslint:recommended}, para realizar el formato automático en los archivos.


\clearpage
\section{Diseño}

Conforme a la metodología de desarrollo centrada en el usuario utilizada en este sistema, se decidió empezar por el desarrollo de un prototipo de alta fidelidad para el diseño del sistema, para posteriormente continuar con el diseño de la infraestructura y los sistemas necesarios para entregar el proyecto, con el objetivo de crear un sistema integral que pudiera satisfacer las necesidades del LCCA al mismo tiempo que cumpliera con los requisitos técnicos necesarios para cumplir los objetivos propuestos.

\subsection{Prototipado de la interfaz gráfica}

Para el desarrollo de un prototipo rápido de la interfaz gráfica, se utilizó el software \textit{Figma}, esto debido a que es un software gratuito, sencillo de utilizar orientado al prototipado de interfaces de alta fidelidad. \textit{Figma} ofrece la capacidad de integrar diferentes \textit{frameworks} de diseño de interfaces, para crear un \textit{design system} (un sistema de diseño) coherente, consistente a través de todo un proyecto. Utilizando este software se decidió utilizar una plantilla de un tablero de control para una plataforma genérica.

Los colores utilizados para el desarrollo de los prototipos, y posteriormente para el desarrollo de la totalidad del proyecto fueron; el color verde \texttt{\#09AB5D} como color principal de la interfaz, debido a su relación con el diseño actual del sitio de consulta de la información de las estaciones meteorológicas, y el color azul \texttt{\#16B2D4} como secundario por su contraste con el color verde y por su actual uso en el sitio existente del LCCA.

Utilizando esta plantilla como base para el lenguaje de diseño de la aplicación, se realizó un prototipado de la interfaz propuesta para evaluar la posible utilidad de la misma, creando un prototipo inicial de alta fidelidad. La primer interfaz en prototiparse fue el tablero de control, después de algunas iteraciones con cambios menores, el prototipo quedó tal como en la Figura \ref{fig:prototype_main_interface}, cabe notar que esta interfaz gráfica tiene elementos en la barra lateral que no corresponden con el proyecto (tal como la sección de chat y otros similares), esto debido a que se dejaron ciertos elementos predefinidos de la plantilla original, para no romper con la estética del diseño.

\begin{figure}[!ht]
	\centering
	\includegraphics[width=1\linewidth]{images/diagrams/0.0.0_Main_interface.png}
	\caption{Prototipo del tablero de control.}
	\label{fig:prototype_main_interface}
\end{figure}

En este prototipo de interfaz se tomaron en cuenta en varios factores, uno de ellos, es que una estación que es monitoreada puede tener un error de conexión que no permita el acceso a la misma, y que al encontrarse con un error de conexión, no es posible obtener el estado de los servicios. De ser así, la información de estos servicios no se muestra. También se tomó en cuenta el identificar las estaciones por un nombre particular, pero también tener presente la dirección IP de la misma (en caso de poseer una) para volver más sencillo el acceso técnico a la información en caso de ser necesario para un usuario técnico. Además, se hizo la consideración de tener una variedad de diferentes servicios que se podrían monitorear, y que el estado de los servicios y de las estaciones fuera independiente.

\pagebreak

La segunda interfaz que fué elegida para su prototipado, fué la interfaz para agregar una nueva estación meteorológica al sistema. Esta interfaz contiene la información elemental que se requiere para registrar una nueva estación. En este caso, se compone de una dirección IP, un usuario, el método de conexión a la misma, el tipo de estación y el dispositivo por el que se accede a la misma. El prototipo realizado se puede observar en la Figura \ref{fig:prototype_add_station}

\begin{figure}[!ht]
	\centering
	\includegraphics[width=1\linewidth]{images/diagrams/1.0.0_Stations_add.png}
	\caption{Prototipo de la interfaz para agregar una nueva estación.}
	\label{fig:prototype_add_station}
\end{figure}

\pagebreak

Posteriormente se creó el componente de la tercera parte importante del monitoreo de las estaciones meteoreológicas, el prototipo de la interfaz de solución de problemas de las estaciones. Al ser un objetivo secundario importante el poder tener y acceder a la información que las estaciones meteorológicas generan, se considera igualmente importante el poder capturar una razón de la solución de los inconvenientes para futuros análisis. Esta información será capturada con la ayuda de una interfaz como la de la Figura \ref{fig:prototype_solve_error}.

\begin{figure}[!ht]
	\centering
	\includegraphics[width=1\linewidth]{images/diagrams/0.1.1_Station_Failing.png}
	\caption{Prototipo de la interfaz de solución de errores.}
	\label{fig:prototype_solve_error}
\end{figure}

\subsection{Diseño de base de datos}

Para el diseño de base de datos del sistema se consideraron dos elementos como relevantes, los datos de las estaciones meteorológicas, que incluyen datos como la conexión a las estaciones y la forma en la que se accederá a ellas, y los eventos que las mismas generen.

Si bien es posible almacenar la información de las estaciones meteorológicas en una \textit{time series database}, en la que la información del estado de cada uno de los servicios y elementos que se monitorean es almacenado sin importar el estado de los mismos, no se considera necesaria la información de los sistemas en su estado funcional, esto debido a que el volumen de datos generado, \textit{O(n*t*s)}, donde \textit{n} es el número de estaciones, \textit{t} la cantidad de veces que la inforamción se consulta por hora y \textit{s} el número de servicios que se consultan, es más grande que la utilidad que se le piensa dar a la información.

La información más relevante que se puede obtener de acuerdo a los objetivos de este proyecto es el estado de las estaciones de monitoreo en caso de falla, así como estimados de la calidad y accesibilidad de los datos que las estaciones recolectan.

Debido a esto, la base de datos se diseñó con el objetivo de almacenar la información en forma de eventos, y tomando en cuenta la utilidad futura de auditoría se decidieron agregar tiempos de creación de eventos y de la solución de los mismos. De la misma forma, el borrado de información no está contemplado en el sistema, para esto se implementó la metodología de borrado de información por medio de \textit{borrados suaves}, los cuales desactivan los registros lógicamente en el sistema sin borrar de forma física los datos de la base de datos.

En cuanto al manejo de la información adicional de las estaciones meteorológicas, es decir, la información de las mismas que no es indispensable para el funcionamiento del sistema pero es necesaria para controles internos, se agregó una tabla de atributos extra. Esta tabla contiene información diversas de las estaciones, y consiste en la forma \textit{llave -> valor} para los campos, esto, para asegurar la mayor flexibilidad posible de los datos y su almacenamiento. Sacrificando velocidades de indexamiento y procesamiento por una mayor libertad para extender y modificar el sistema.

Con todas estas consideraciones en mente, se creó una base de datos que corresponde con los contenidos que se muestran en el modelo entidad-relación que se puede obeservar en la Figura \ref{fig:diagrama_base_de_datos}.

\begin{figure}[!ht]
	\centering
	\includegraphics[width=0.86\linewidth]{images/diagrams/database_diagram.png}
	\caption{Modelo entidad-relación del proyecto meteoreo}
	\label{fig:diagrama_base_de_datos}
\end{figure}

\subsection{Arquitectura del sistema}

Para realizar la conexión a las estaciones meteorológicas, se decidió dividir el proyecto en dos componentes principales, un módulo de generación de reportes y un sistema de controladores que contuvieran el código de conexión y restauración de reportes de las estaciones.

Los drivers se desarrollaron con el objetivo de tener una plataforma estándar para la consulta de los datos de las estaciones.

Tomando como referencia el proyecto de \textit{Monitoring Plugins} \cite{monitoring_plugins}, el cual es compatible con diversos proyectos especializados en monitoreo de sistemas de alta resilencia, tales como \textit{Nagios} y \textit{Icinga}, se decidió crear un proyecto basado en drivers fácilmente extendibles.

%! TODO: Todo el desarrollo de la arquitectura del sistema

\subsection{Selección del motor de base de datos}

Para el caso de uso del centro de monitoreo de estaciones meteorológicas de la UACJ, en el que la red actual cuenta con 13 estaciones, no es necesario considerar como cuello de botella el motor de base de datos que se utilizará para el sistema. Esto debido a que, con un tiempo mínimo para la consulta del estado de las estaciones de hasta 5 minutos entre consultas, el sistema podría funcionar incluso con un tiempo promedio de 23 segundos desde la consulta hasta el almacenamiento de la información. Esto, sin tomar en cuenta que es posible paralelizar el proceso de consulta y generación de eventos de las estaciones meteorológicas, por lo que no se considera como algo relevante la selección de un motor de base de datos que cuente con alto rendimiento de lectura y/o escritura de la información.

Debido a que la infraestructura del sistema de las estaciones meteorológicas ya utiliza un motor relacional de base de datos adecuado para el proyecto, MySQL, se pretende utilizarlo para este proyecto, reduciendo la carga de mantenimiento para el equipo de la universidad, además de un sistema familiar que permitirá a los involucrados realizar consultas a la información sin necesidad de aprender nuevas tecnologías.

Para esto, se utilizó la flexibilidad que ofrecen los sistemas modelado de objetos y roles (ORM, por sus siglas en inglés) \cite{Halpin2006}, en la que se permite el crear sistemas agnósticos de un motor de base de datos en específico, y la creación de modelos, esquemas y relaciones de base de datos se dejan al \textit{framework} de modelado de datos. Esto además ofrece soporte para migraciones para realizar actualizaciones de base de datos controladas en caso de requerir extender un sistema existente.

El motor de base de datos seleccionado para el desarrollo local del proyecto fué el conocido como \textit{SQLite}, debido a la flexibilidad que ofrece al ser una base de datos que sólo depende de un archivo para funcionar y que no requiere de instalar paquetes de software extra en la estación que se utiliza para desarrollar y probar el proyecto.

\subsection{Configuración de ambiente de
desarrollo}\label{configuraciuxf3n-de-ambiente-de-desarrollo}

El ambiente de desarrollo que se seleccionó para la sección de python del proyecto fué seleccionado con el objetivo de proveer la mayor flexibilidad y portabilidad a corto y largo plazo, haciendo fácil la modificación posterior del proyecto por los que no estuvieron involucrados inicialmente, y sencillo de replicar para futuras investigaciones. Para lograr estos objetivos, se optó por realizar el proyecto con ayuda de las tecnologías de \textit{Docker}, debido a que permite realizar imágenes de proyectos de forma sencilla, y permite hacer contenedores multiplataforma que con una mínima configuración se vuelven útiles para el desarrollo.

% Con la finalidad de tener un contenedor de desarrollo que pueda ser replicado con la mínima configuración se eligió la plataforma docker, por su amplia adopción y por las facilidades que ofrece para crear sistemas complejos que dependen de varios servicios sin tener que realizar configuraciones en el sistema que puedan ser perdidas al momento de cambiar a otro.

Debido a la sencillez que el sistema de cnfiguración de contenedores \emph{docker compose} ofrece, se eligió para almacenar los parámetros de configuración de los contenedores en vez de crear comandos compatibles con docker para ello. Esto permite una fácil edición de los servicios y la aplicación de los mismos de una forma estandarizada que permite una comprensión más eficaz de los parámetros y de las dependencias.

El archivo de configuración fué almacenado en la raíz del proyecto, con el nombre de \texttt{docker-compose.yml}  tal como el estándar de la utilería \texttt{docker-compose} sugiere, y este archivo tiene el contenido siguiente que se muestra en el listado \ref{lst:docker-compose}. En este archivo, se especifica que se requiere de un servicio de \textit{MySQL}, el cuál fué utilizado para corroborar que el desarrollo era de utilidad con el motor seleccionado para producción, así como una referencia al archivo \textit{Docker} en el que se tiene el contenedor que se utilizará para el desarrollo. Además, se hace referencia a algunas variables, como \texttt{MYSQL\_DATABASE}, que son obtenidas de un archivo \texttt{\.env} estandarizado en la raíz del proyecto.

%! TODO: INSERTAR Referencia de archivo docker-compose en el anterior parrafito
% \begin{listing}
\begin{minted}{yaml}
version: '3.3'
services:
  api:
    container_name: meteoreo-api
    build:
      context: ./
      dockerfile: Dockerfile
    volumes:
      - './:/app:delegated'
    depends_on:
      - mysql
    environment:
      - WEB_CONCURRENCY=2
      - PORT=80
      - PRE_START_PATH=/app/app/prestart.sh
      - GUNICORN_CMD_ARGS="--reload"
    ports:
      - '81:80'
    networks:
      - meteoreo-backend

  mysql:
    image: mysql
    container_name: meteoreo-mysql
    environment:
      MYSQL_DATABASE: '${MYSQL_DATABASE}'
      MYSQL_ROOT_PASSWORD: '${MYSQL_ROOT_PASSWORD}'
      MYSQL_PASSWORD: '${MYSQL_PASSWORD}'
      MYSQL_USER: '${MYSQL_USER}'
      SERVICE_TAGS: dev
      SERVICE_NAME: mysql
    ports:
      - '3306:3306'
    networks:
      - meteoreo-backend

networks:
  meteoreo-backend:
    driver: bridge
\end{minted}
% \caption{Archivo docker-compose}
% \label{lst:docker-compose}
% \end{listing}

El archivo de \textit{docker-compose} tiene una dependencia con un archivo de Docker, que se pretende que facilite la adición de librerías adicionales al proyecto en la posteridad. Actualmente, extiende la imagen existente de \textit{tiangolo}, el proyecto \textit{Uvicorn-Gunicorn-Fastapi}. Esta imagen fue utilizada como base debido a su increíble flexibilidad para el desarrollo de proyectos en FastApi, sus optimizaciones automáticas para el balanceo de cargas entre diversos procesos creados automáticamente (ya que python es monoproceso) y además, por ser una imagen altamente mantenida por la comunidad, debido a su popularidad. En este archivo también se especifíca el instalar la librería \texttt{inteutils-ping} debido a que el proyecto dependerá de realizar pruebas por \texttt{ping} para revisar la conectividad con las estaciones antes de intentar realizar una conexión y la imagen base no tenía esta librería. \ref{lst:dockerfile}.

%! TODO: Insertar referencia de dockerfile en el anterior parrafito

\begin{listing}[h]
\begin{minted}{dockerfile}
FROM tiangolo/uvicorn-gunicorn-fastapi:python3.7

# Installs lib to do pings from the server
RUN apt-get update && apt-get install -y \
   inetutils-ping \
   && rm -rf /var/lib/apt/lists/*

CMD [ "/start-reload.sh" ]
\end{minted}
\caption[Dockerfile]{Archivo Dockerfile}
\label{lst:dockerfile}
\end{listing}

El editor de código seleccionado para el desarrollo del proyecto es Visual Studio Code, el cual posee una gran extensibilidad y predeterminados sensibles que permiten configurar el ambiente de trabajo de la forma que más sea conveniente para el desarrollo del proyecto, además provee la funcionalidad de \textit{devcontainers}, los cuales son parte de una extensión que permiten el crear ambientes de desarrollo dentro de ambientes virtuales en docker, utilizando las herramientas instaladas en la imagen de docker y que no requieren de configuración adicional por parte del desarrollador para comenzar a trabajar en un proyecto. Al detectar un archivo \texttt{devcontainer.json}, esta extensión automáticamente informa al desarrollador de su existencia y le invita a iniciar su ambiente de desarrollo utilizando los parámetros definidos en el arhivo.

En este archivo se especifica un nombre para identificar el ambiente de desarrollo que sea reconocible por el desarrollador, la localización del archivo que describe el contenedor, y una lista de extensiones para el editor de código. Entre las más importantes se encuentra \emph{pylance} que permite  realizar el formato automático de códgo con pep8 y \emph{magicpython} una adición al editor de código que provee un motor de autocompletación para python, las demás siendo preferencias personales útiles para agilizar el desarrollo del proyecto.

%! TODO: Insertar referencia del archivo devcontainer en el parrafito anterior

\begin{minted}{json}
{
  "name": "Meteoreo API",
  "service": "api",
  "remoteUser": "root",
  "shutdownAction": "stopCompose",
  "workspaceFolder": "/app",
  "dockerComposeFile": "../docker-compose.yml",
  "extensions": [
    "editorconfig.editorconfig",
    "mikestead.dotenv",
    "njpwerner.autodocstring",
    "aaron-bond.better-comments",
    "mhutchie.git-graph",
    "hookyqr.beautify",
    "magicstack.magicpython",
    "gruntfuggly.todo-tree",
    "ms-python.vscode-pylance",
    "sleistner.vscode-fileutils"
  ]
}
\end{minted}

Para el desarrollo de la sección de la interfaz de web del proyecto, se instaló en la máquina de desarrollo NPM versión 14.8.1, debido a que era la última versión \textit{LTS} (Soporte a largo plazo, por sus siglas en inglés) disponible, y el sistema de manejo de dependencias \textit{yarn} debido a las ventajas que ofrece sobre \textit{npm}, tales como mayor velocidad de instalación de paquetes y caché multiproyecto. No se vió como un elemento necesario el intergrar Docker o algún otro tipo de tecnología de contenedores para el proyecto de frontend, debido a la ubicuidad de las herramientas y la simpleza de instalación y de mantenimiento de las mismas.


\clearpage

\section{Desarrollo}

Después de haber realizado el análisis inicial de el alcance del proyecto y las necesidades de los usuarios, se comenzó con el desarrollo del proyecto. Este desarrollo se hizo en tres partes. Primero, el desarrollo de un módulo de monitoreo de las estaciones meteorológicas por medio de drivers,  después un API como intermediaria entre la información almacenada en la base de datos y una interfaz gráfica para el monitoreo eficaz.

\subsection{Del módulo de monitoreo de estaciones}

El módulo de monitoreo de estaciones meteorológicas tiene como objetivo el observar la información obtenida por los diversos drivers de conexión a las estaciones meteorológicas y generar reportes conforme sea necesario, la lógica de reporte es tal como se muestra en la figura \ref{fig:logica_de_reporte}.

%! Agregar acciones al diagrama
%! Cambiar por diagrama de secuencia
% Agregar un fin después de generar la alerta
% debe hber un solo inicio y final

\begin{figure}[!ht]
	\centering
	\includegraphics[width=1\linewidth]{images/diagrams/report_logic.png}
	\caption{Lógica de reporte del estado de las estaciones meteorológicas}
	\label{fig:logica_de_reporte}
\end{figure}


\subsection{Del API para el acceso a la información}

\subsection{De la interfaz gráfica del proyecto}

PAra el desarrollo del proyecto, se utilizó tailwind.

\section{Avances}


\section{Módulo de monitoreo}

\subsection*{Requisitos}

\subsection*{Seguridad}

\subsection*{Método de conexion}


\chapter{Resultados y Discusiones}
[En este capítulo se presentan los resultados obtenidos correspondientes al proyecto descrito en el capítulo anterior. Los resultados se pueden presentar en tablas o gráfica y deben ser redactados y organizados de tal manera que sea fácil de comprender por los lectores.

La los resultados no se explican por si mismo, por lo que es necesario una discusión que los explique y muestre cómo ayudan a resolver el problema definido en el capítulo 1. La discusión puede mencionar someramente los resultados antes de discutirlos, pero no debe repetirlos en detalle. No prolongues la discusión citando trabajos ``relacionados'' o planteando explicaciones poco probables. Ambas acciones distraen al lector y lo alejan de la discusión realmente importante. La discusión puede incluir recomendaciones y sugerencias para investigaciones futuras, tales como métodos alternos que podrían dar mejores resultados, tareas que no se hicieron y que en retrospectiva debieron hacerse, y aspectos que merecen explorarse en las próximas investigaciones.] 





\chapter{Conclusiones}\label{conclusiones}
% [Estos son los enunciados más importantes y más fuertes que se deben realizar acerca de los resultados y discusiones. Las conclusiones deben resumir el contenido y el propósito del proyecto. Se debe hacer énfasis en lo que queremos que se recuerde acerca del proyecto y realizar una síntesis de los resultados que se derivaron de los objetivos específicos trazados inicialmente (y que dieron respuesta a las preguntas de investigación si es que existen). No deben aparecer elementos nuevos o que no fueron discutidos, por ejemplo nuevos resultados observados en otros trabajos. Las conclusiones se refieren única y exclusivamente al proyecto desarrollado.

En este documento se reporta el desarrollo de un sistema de monitoreo y control para estaciones meteorológicas. A través de este apartado se presentan las conclusiones a las que se llegó durante el desarrollo del proyecto, así como las recomendaciones para trabajos futuros.

%* proyecto para titularse, decir que se aplicó lo aprendido, decir cuestiones diversas.

\section{Con respecto al objetivo de la investigación}

% Desarrollar un sistema de monitoreo y control para estaciones meteorológicas que permita un monitoreo contínuo de las mismas por parte de personal no especializado, con el objetivo de proveer un mejor mantenimiento preventivo y correctivo de las estaciones meteorológicas y de calidad del aire.

%* Objetivo general entre comillas, cita textual.

Con respecto al objetivo general del proyecto, que es el desarrollo de un sistema que permita un monitoreo contínuo de estaciones meteorológicas, para proveer un mejor mantenimiento preventivo y correctivo, se logró de manera exitosa y funcional en resolver el problema descrito en la sección 1.2, conforme a los resultados presentados en el capítulo 4.


Con la realización del proyecto se lograron los siguientes resultados:

%? Desarrollar un sistema central para coordinar y recolectar los datos del estado de las estaciones.

%? Diseñar un API REST para consultar el estado de las estaciones meteorológicas.

%? Diseñar e implementar una interfaz gráfica para monitorear el estatus de las estaciones.

%? Integrar las diferentes estaciones meteorológicas existentes al sistema creando los controladores correspondientes.

%? Integrar un sistema existente de notificaciones/alertas de terceros para fallos críticos de las estaciones.
\begin{itemize}
   \item Se desarrolló un sistema para recolectar los datos del estado de las estaciones, que permitió un mejor control de la información y estado de las mismas.
   %* \item Con una recolección adecuada de los datos de la sestaciones permiten un mejor control del estado de las mismas.

   \item Se diseñó un API REST para facilitar la consulta de los datos del estado de las estaciones, documentada con OpenAPI de forma automática y,

   %* EL API REST disenado facilita la consulta del estado de las estaciones. (Se puede hacer la referencia al capítulo 3)

   \item Se generó una interfaz gráfica para la consulta del estado de las estaciones, que permitió una mejor visualización de los datos, integrado con un sistema de notificación de alertas.

   %* La interfaz gráfica permite la visualización de los datos de las estaciones de una manera más visual y sencilla.

\end{itemize}



\section{Con respecto al futuro del proyecto}

Por cuestiones de que las siguientes funcionalidades se encuentran fuera del alcance de este proyecto, se recomienda que sean implementadas posteriormente:

\begin{itemize}
   \item El utilizar un sistema automatizado para la solución de incidentes de fallos en las estaciones meteorológicas.
   \item Generar reportes automatizados de calidad de datos y de líneas de tiempo de las estaciones con la información recolectada.
   \item El desarrollo de un sistema de predicción de tendencias de errores, que ayude a la toma de decisiones de mantenimiento preventivo y correctivo.
\end{itemize}

Se recomienda el automatizar



Tomar los componentes que fueron creados con tailwind y vue y realmente hacer uso de lasa clases de componentes e información.





% [La forma más simple de presentar las conclusiones es enumerándolas consecutivamente, pero podrías optar por recapitular brevemente el contenido del artículo, mencionando someramente su propósito, los métodos principales, los datos más sobresalientes y la contribución más importante de la investigación. La sección de conclusiones no debe repetir innecesariamente el contenido del resumen.]

\section{Recomendaciones para futuras investigaciones}

Debido a que las siguientes

%! Recomendaciones al laboratorio para cuando la infraestructura actual quede descontinuada.

% Fin de Conclusiones



\bibliographystyle{ieeetr}
\bibliography{referencias}


\appendix   % inician los apÈndices de la tesis

% los capÌtulos que se incluyan a partir de aquÌ aparecen
% como apÈndices
% % Inicio del ApÈndice A
\chapter{Nombre del Apéndice}\label{apendiceA}

[Sustituye este texto.
En esta sección opcional se deberá incluir información secundaria o material importante que es muy extenso. El apéndice se coloca después de la literatura citada. Ejemplos de información que puede colocarse en el apéndice: una lista de universidades visitadas; los datos obtenidos de todas las repeticiones del experimento; derivaciones matemáticas extensas; todos los resultados del análisis estadístico (incluyendo quizás los no significativos) y mapas de distribución para cada fenómeno estudiado; listados completos de código fuente; etc.]



% estos comandos generan la bilbiografÌa
% La bibliografÌa se obtiene de la base de datos
% Estilos:
%	 plain (sistema numÈrico, orden alfabÈtico)
%	 unsrt (Sistema numÈrico, en el orden en que van apareciendo las citas) ------
%	 alpha (Sistema autor-fecha abreviado, orden alfabÈtico)
%	 abbrv (Sistema autor-fecha abreviado, estilo bibliogr·fico alfabÈtico abreviado)
%	 ieeepes (Bibliography Style file for articles according to IEEE instructions) Basado en unsrt



\end{document}
